% Options for packages loaded elsewhere
\PassOptionsToPackage{unicode}{hyperref}
\PassOptionsToPackage{hyphens}{url}
%
\documentclass[
]{article}
\usepackage{amsmath,amssymb}
\usepackage{lmodern}
\usepackage{iftex}
\ifPDFTeX
  \usepackage[T1]{fontenc}
  \usepackage[utf8]{inputenc}
  \usepackage{textcomp} % provide euro and other symbols
\else % if luatex or xetex
  \usepackage{unicode-math}
  \defaultfontfeatures{Scale=MatchLowercase}
  \defaultfontfeatures[\rmfamily]{Ligatures=TeX,Scale=1}
\fi
% Use upquote if available, for straight quotes in verbatim environments
\IfFileExists{upquote.sty}{\usepackage{upquote}}{}
\IfFileExists{microtype.sty}{% use microtype if available
  \usepackage[]{microtype}
  \UseMicrotypeSet[protrusion]{basicmath} % disable protrusion for tt fonts
}{}
\makeatletter
\@ifundefined{KOMAClassName}{% if non-KOMA class
  \IfFileExists{parskip.sty}{%
    \usepackage{parskip}
  }{% else
    \setlength{\parindent}{0pt}
    \setlength{\parskip}{6pt plus 2pt minus 1pt}}
}{% if KOMA class
  \KOMAoptions{parskip=half}}
\makeatother
\usepackage{xcolor}
\usepackage[margin=1in]{geometry}
\usepackage{color}
\usepackage{fancyvrb}
\newcommand{\VerbBar}{|}
\newcommand{\VERB}{\Verb[commandchars=\\\{\}]}
\DefineVerbatimEnvironment{Highlighting}{Verbatim}{commandchars=\\\{\}}
% Add ',fontsize=\small' for more characters per line
\usepackage{framed}
\definecolor{shadecolor}{RGB}{248,248,248}
\newenvironment{Shaded}{\begin{snugshade}}{\end{snugshade}}
\newcommand{\AlertTok}[1]{\textcolor[rgb]{0.94,0.16,0.16}{#1}}
\newcommand{\AnnotationTok}[1]{\textcolor[rgb]{0.56,0.35,0.01}{\textbf{\textit{#1}}}}
\newcommand{\AttributeTok}[1]{\textcolor[rgb]{0.77,0.63,0.00}{#1}}
\newcommand{\BaseNTok}[1]{\textcolor[rgb]{0.00,0.00,0.81}{#1}}
\newcommand{\BuiltInTok}[1]{#1}
\newcommand{\CharTok}[1]{\textcolor[rgb]{0.31,0.60,0.02}{#1}}
\newcommand{\CommentTok}[1]{\textcolor[rgb]{0.56,0.35,0.01}{\textit{#1}}}
\newcommand{\CommentVarTok}[1]{\textcolor[rgb]{0.56,0.35,0.01}{\textbf{\textit{#1}}}}
\newcommand{\ConstantTok}[1]{\textcolor[rgb]{0.00,0.00,0.00}{#1}}
\newcommand{\ControlFlowTok}[1]{\textcolor[rgb]{0.13,0.29,0.53}{\textbf{#1}}}
\newcommand{\DataTypeTok}[1]{\textcolor[rgb]{0.13,0.29,0.53}{#1}}
\newcommand{\DecValTok}[1]{\textcolor[rgb]{0.00,0.00,0.81}{#1}}
\newcommand{\DocumentationTok}[1]{\textcolor[rgb]{0.56,0.35,0.01}{\textbf{\textit{#1}}}}
\newcommand{\ErrorTok}[1]{\textcolor[rgb]{0.64,0.00,0.00}{\textbf{#1}}}
\newcommand{\ExtensionTok}[1]{#1}
\newcommand{\FloatTok}[1]{\textcolor[rgb]{0.00,0.00,0.81}{#1}}
\newcommand{\FunctionTok}[1]{\textcolor[rgb]{0.00,0.00,0.00}{#1}}
\newcommand{\ImportTok}[1]{#1}
\newcommand{\InformationTok}[1]{\textcolor[rgb]{0.56,0.35,0.01}{\textbf{\textit{#1}}}}
\newcommand{\KeywordTok}[1]{\textcolor[rgb]{0.13,0.29,0.53}{\textbf{#1}}}
\newcommand{\NormalTok}[1]{#1}
\newcommand{\OperatorTok}[1]{\textcolor[rgb]{0.81,0.36,0.00}{\textbf{#1}}}
\newcommand{\OtherTok}[1]{\textcolor[rgb]{0.56,0.35,0.01}{#1}}
\newcommand{\PreprocessorTok}[1]{\textcolor[rgb]{0.56,0.35,0.01}{\textit{#1}}}
\newcommand{\RegionMarkerTok}[1]{#1}
\newcommand{\SpecialCharTok}[1]{\textcolor[rgb]{0.00,0.00,0.00}{#1}}
\newcommand{\SpecialStringTok}[1]{\textcolor[rgb]{0.31,0.60,0.02}{#1}}
\newcommand{\StringTok}[1]{\textcolor[rgb]{0.31,0.60,0.02}{#1}}
\newcommand{\VariableTok}[1]{\textcolor[rgb]{0.00,0.00,0.00}{#1}}
\newcommand{\VerbatimStringTok}[1]{\textcolor[rgb]{0.31,0.60,0.02}{#1}}
\newcommand{\WarningTok}[1]{\textcolor[rgb]{0.56,0.35,0.01}{\textbf{\textit{#1}}}}
\usepackage{graphicx}
\makeatletter
\def\maxwidth{\ifdim\Gin@nat@width>\linewidth\linewidth\else\Gin@nat@width\fi}
\def\maxheight{\ifdim\Gin@nat@height>\textheight\textheight\else\Gin@nat@height\fi}
\makeatother
% Scale images if necessary, so that they will not overflow the page
% margins by default, and it is still possible to overwrite the defaults
% using explicit options in \includegraphics[width, height, ...]{}
\setkeys{Gin}{width=\maxwidth,height=\maxheight,keepaspectratio}
% Set default figure placement to htbp
\makeatletter
\def\fps@figure{htbp}
\makeatother
\setlength{\emergencystretch}{3em} % prevent overfull lines
\providecommand{\tightlist}{%
  \setlength{\itemsep}{0pt}\setlength{\parskip}{0pt}}
\setcounter{secnumdepth}{-\maxdimen} % remove section numbering
\ifLuaTeX
  \usepackage{selnolig}  % disable illegal ligatures
\fi
\IfFileExists{bookmark.sty}{\usepackage{bookmark}}{\usepackage{hyperref}}
\IfFileExists{xurl.sty}{\usepackage{xurl}}{} % add URL line breaks if available
\urlstyle{same} % disable monospaced font for URLs
\hypersetup{
  pdftitle={Can scale-dependant geodiversity improve species distribution models in a Neotropical biodiversity hotspot?},
  pdfauthor={Beth Gerstner},
  hidelinks,
  pdfcreator={LaTeX via pandoc}}

\title{Can scale-dependant geodiversity improve species distribution
models in a Neotropical biodiversity hotspot?}
\author{Beth Gerstner}
\date{2023-01-02}

\begin{document}
\maketitle

Overview: This script obtains the occurrence records for species of
interest in Colombia. It extracts records over the BioModelos API.
Records are then counted. Certain species could not be downloaded over
the API and had to be obtained manually through the BioModelos website.
These are then cleaned and merged into the full dataset.

*note currently this API does not seem to be working.

Data output: csv file of occurrence records for all species.

Date: 10/1/22 Modified: 1/1/23

Load packages

\begin{Shaded}
\begin{Highlighting}[]
\FunctionTok{library}\NormalTok{(spocc)}
\FunctionTok{library}\NormalTok{(spThin)}
\FunctionTok{library}\NormalTok{(dismo)}
\FunctionTok{library}\NormalTok{(rgeos)}
\FunctionTok{library}\NormalTok{(ENMeval)}
\FunctionTok{library}\NormalTok{(wallace)}
\FunctionTok{library}\NormalTok{(dplyr)                                                }
\FunctionTok{library}\NormalTok{(plyr)                                                 }
\FunctionTok{library}\NormalTok{(readr)  }
\FunctionTok{library}\NormalTok{(data.table)}
\FunctionTok{library}\NormalTok{(janitor)}
\end{Highlighting}
\end{Shaded}

Read in species list of interest. These are species that have validated
models on BioModelos. We use these occurrence records because they have
been vetted by experts at BioModelos. Also, we would like to make
comparisons with BioModelos models as a way of validating our method of
making distribution maps.

\begin{Shaded}
\begin{Highlighting}[]
\NormalTok{scientific\_names\_mammals }\OtherTok{\textless{}{-}} \FunctionTok{read.csv}\NormalTok{(}\StringTok{"C:/Users/bgers/Desktop/final\_species\_list.csv"}\NormalTok{)}
\NormalTok{bioKey }\OtherTok{\textless{}{-}} \StringTok{"5NbMdjylEPEN1cBCIsX9dl:3vbVvDAXQdjf40NeYHfN1g"}
\end{Highlighting}
\end{Shaded}

Download occurrence records from BioModelos

\begin{Shaded}
\begin{Highlighting}[]
\NormalTok{all.mammals.biomod }\OtherTok{\textless{}{-}} \FunctionTok{data.frame}\NormalTok{()}
\ControlFlowTok{for}\NormalTok{(i }\ControlFlowTok{in} \DecValTok{1}\SpecialCharTok{:}\FunctionTok{nrow}\NormalTok{(scientific\_names\_mammals))\{}
\NormalTok{  species.i }\OtherTok{\textless{}{-}}\NormalTok{ scientific\_names\_mammals[i,}\StringTok{"species"}\NormalTok{]}
\NormalTok{  bioModelos }\OtherTok{\textless{}{-}} \FunctionTok{occs\_biomodelos}\NormalTok{(}
    \AttributeTok{spN =}\NormalTok{ species.i,}
    \AttributeTok{bioKey =}\NormalTok{ bioKey)}
\NormalTok{  occs\_i }\OtherTok{\textless{}{-}}\NormalTok{ bioModelos}\SpecialCharTok{$}\NormalTok{cleaned}
\NormalTok{  occs\_i }\OtherTok{\textless{}{-}}\NormalTok{ occs\_i[,}\FunctionTok{c}\NormalTok{(}\StringTok{"scientific\_name"}\NormalTok{,}\StringTok{"latitude"}\NormalTok{,}\StringTok{"longitude"}\NormalTok{,}\StringTok{"year"}\NormalTok{,}\StringTok{"state\_province"}\NormalTok{,}\StringTok{"record\_type"}\NormalTok{, }\StringTok{"catalog\_number"}\NormalTok{, }\StringTok{"institution\_code"}\NormalTok{)]}
\NormalTok{  all.mammals.biomod}\OtherTok{=} \FunctionTok{rbind}\NormalTok{(all.mammals.biomod, occs\_i)}
\NormalTok{\}}
\end{Highlighting}
\end{Shaded}

Split species list because one species Cheracebus lucifer was causing
the loop to throw errors.

\begin{Shaded}
\begin{Highlighting}[]
\NormalTok{all.mammals}\FloatTok{.1} \OtherTok{\textless{}{-}} \FunctionTok{data.frame}\NormalTok{()}
\ControlFlowTok{for}\NormalTok{(i }\ControlFlowTok{in} \DecValTok{27}\SpecialCharTok{:}\DecValTok{32}\NormalTok{)\{}
\NormalTok{  species.i }\OtherTok{\textless{}{-}}\NormalTok{ scientfic\_names\_mammals[i,}\StringTok{"species"}\NormalTok{]}
\NormalTok{  bioModelos }\OtherTok{\textless{}{-}} \FunctionTok{occs\_biomodelos}\NormalTok{(}
    \AttributeTok{spN =}\NormalTok{ species.i,}
    \AttributeTok{bioKey =}\NormalTok{ bioKey)}
\NormalTok{  occs\_i }\OtherTok{\textless{}{-}}\NormalTok{ bioModelos}\SpecialCharTok{$}\NormalTok{cleaned}
\NormalTok{  occs\_i }\OtherTok{\textless{}{-}}\NormalTok{ occs\_i[,}\FunctionTok{c}\NormalTok{(}\StringTok{"scientific\_name"}\NormalTok{,}\StringTok{"latitude"}\NormalTok{,}\StringTok{"longitude"}\NormalTok{,}\StringTok{"year"}\NormalTok{,}\StringTok{"state\_province"}\NormalTok{,}\StringTok{"record\_type"}\NormalTok{, }\StringTok{"catalog\_number"}\NormalTok{, }\StringTok{"institution\_code"}\NormalTok{)]}
\NormalTok{  all.mammals}\FloatTok{.1}\OtherTok{=} \FunctionTok{rbind}\NormalTok{(all.mammals, occs\_i)}
\NormalTok{\}}
\end{Highlighting}
\end{Shaded}

Join list for species downloaded over API

\begin{Shaded}
\begin{Highlighting}[]
\NormalTok{all.mammals.full }\OtherTok{\textless{}{-}} \FunctionTok{rbind}\NormalTok{(all.mammals, all.mammals}\FloatTok{.1}\NormalTok{)}
\FunctionTok{setwd}\NormalTok{(}\StringTok{"C:/Users/bgers/Desktop/MSU/Zarnetske\_Lab/Data/Chapter\_1/candidate\_species\_dataset"}\NormalTok{)}

\CommentTok{\#write to csv}
\FunctionTok{write.csv}\NormalTok{(all.mammals.full,}\AttributeTok{path=}\StringTok{"all\_mammals\_biomodelos.csv"}\NormalTok{)}
\end{Highlighting}
\end{Shaded}

Count number of records per species

\begin{Shaded}
\begin{Highlighting}[]
\NormalTok{species\_num }\OtherTok{\textless{}{-}}\NormalTok{ all.mammals }\SpecialCharTok{\%\textgreater{}\%}
  \FunctionTok{count}\NormalTok{(scientific\_name, }\AttributeTok{sort =} \ConstantTok{TRUE}\NormalTok{) }
\end{Highlighting}
\end{Shaded}

Downloaded certain species individually on the BioModelos website
because of errors and pulled them in all together

\begin{Shaded}
\begin{Highlighting}[]
\NormalTok{single\_data }\OtherTok{\textless{}{-}} \FunctionTok{list.files}\NormalTok{(}\AttributeTok{path =} \StringTok{"C:/Users/bgers/Desktop/MSU/Zarnetske\_Lab/Data/Chapter\_1/candidate\_species\_dataset/biomodelos\_singles"}\NormalTok{,    }
                       \AttributeTok{pattern =} \StringTok{"*.csv"}\NormalTok{, }\AttributeTok{full.names =} \ConstantTok{TRUE}\NormalTok{) }\SpecialCharTok{\%\textgreater{}\%} 
  \FunctionTok{lapply}\NormalTok{(read\_csv) }\SpecialCharTok{\%\textgreater{}\%}                                           
\NormalTok{  bind\_rows                                                      }
\end{Highlighting}
\end{Shaded}

\begin{verbatim}
## Rows: 19 Columns: 28
## -- Column specification --------------------------------------------------------
## Delimiter: ","
## chr (16): suggestedStateProvince, acceptedNameUsage, basisOfRecord, catalogN...
## dbl  (9): taxID, day, environmentalOutlier, month, year, reported, longitude...
## lgl  (3): url, collectionCode, updated
## 
## i Use `spec()` to retrieve the full column specification for this data.
## i Specify the column types or set `show_col_types = FALSE` to quiet this message.
## Rows: 49 Columns: 28
## -- Column specification --------------------------------------------------------
## Delimiter: ","
## chr (16): suggestedStateProvince, acceptedNameUsage, basisOfRecord, catalogN...
## dbl  (9): taxID, day, environmentalOutlier, month, year, reported, longitude...
## lgl  (3): url, collectionCode, updated
## 
## i Use `spec()` to retrieve the full column specification for this data.
## i Specify the column types or set `show_col_types = FALSE` to quiet this message.
## Rows: 38 Columns: 28
## -- Column specification --------------------------------------------------------
## Delimiter: ","
## chr (16): suggestedStateProvince, acceptedNameUsage, basisOfRecord, catalogN...
## dbl  (8): taxID, day, environmentalOutlier, month, year, longitude, latitude...
## lgl  (4): url, collectionCode, reported, updated
## 
## i Use `spec()` to retrieve the full column specification for this data.
## i Specify the column types or set `show_col_types = FALSE` to quiet this message.
## Rows: 27 Columns: 28
## -- Column specification --------------------------------------------------------
## Delimiter: ","
## chr (16): suggestedStateProvince, acceptedNameUsage, basisOfRecord, catalogN...
## dbl  (8): taxID, day, environmentalOutlier, month, year, longitude, latitude...
## lgl  (4): url, collectionCode, reported, updated
## 
## i Use `spec()` to retrieve the full column specification for this data.
## i Specify the column types or set `show_col_types = FALSE` to quiet this message.
## Rows: 26 Columns: 28
## -- Column specification --------------------------------------------------------
## Delimiter: ","
## chr (17): suggestedStateProvince, acceptedNameUsage, suggestedCounty, collec...
## dbl  (8): taxID, day, environmentalOutlier, month, year, longitude, latitude...
## lgl  (3): url, reported, updated
## 
## i Use `spec()` to retrieve the full column specification for this data.
## i Specify the column types or set `show_col_types = FALSE` to quiet this message.
\end{verbatim}

\begin{Shaded}
\begin{Highlighting}[]
\CommentTok{\#Convert to dataframe}
\NormalTok{single\_data }\OtherTok{\textless{}{-}} \FunctionTok{as.data.frame}\NormalTok{(single\_data)}

\CommentTok{\#Subset columns}
\NormalTok{single\_data\_clean }\OtherTok{\textless{}{-}}\NormalTok{ single\_data[,}\FunctionTok{c}\NormalTok{(}\StringTok{"species"}\NormalTok{,}\StringTok{"latitude"}\NormalTok{,}\StringTok{"longitude"}\NormalTok{,}\StringTok{"year"}\NormalTok{, }\StringTok{"stateProvince"}\NormalTok{,}\StringTok{"basisOfRecord"}\NormalTok{,}\StringTok{"catalogNumber"}\NormalTok{,}
                                    \StringTok{"institutionCode"}\NormalTok{)]}
\end{Highlighting}
\end{Shaded}

Read in API dataset if haven't run above \emph{all.mammals \textless-
read.csv(``C:/Users/bgers/Desktop/MSU/Zarnetske\_Lab/Data/Chapter\_1/candidate\_species\_dataset/all\_mammals\_biomodelos\_2022.csv'')}

Rename multiple columns for old to new and bind all data

\begin{Shaded}
\begin{Highlighting}[]
\FunctionTok{setnames}\NormalTok{(single\_data\_clean, }\AttributeTok{old =} \FunctionTok{c}\NormalTok{(}\StringTok{\textquotesingle{}species\textquotesingle{}}\NormalTok{,}\StringTok{\textquotesingle{}stateProvince\textquotesingle{}}\NormalTok{,}\StringTok{\textquotesingle{}basisOfRecord\textquotesingle{}}\NormalTok{,}\StringTok{\textquotesingle{}catalogNumber\textquotesingle{}}\NormalTok{,}\StringTok{\textquotesingle{}institutionCode\textquotesingle{}}\NormalTok{), }
         \AttributeTok{new =} \FunctionTok{c}\NormalTok{(}\StringTok{\textquotesingle{}scientific\_name\textquotesingle{}}\NormalTok{,}\StringTok{\textquotesingle{}state\_province\textquotesingle{}}\NormalTok{,}\StringTok{\textquotesingle{}record\_type\textquotesingle{}}\NormalTok{,}\StringTok{\textquotesingle{}catalog\_number\textquotesingle{}}\NormalTok{,}\StringTok{\textquotesingle{}institution\_code\textquotesingle{}}\NormalTok{))}
\FunctionTok{names}\NormalTok{(single\_data\_clean)}

\CommentTok{\#bind all biomodelos datasets together}
\NormalTok{full\_biomodelos\_data }\OtherTok{\textless{}{-}} \FunctionTok{rbind}\NormalTok{(all.mammals,single\_data\_clean)}
\FunctionTok{write.csv}\NormalTok{(full\_biomodelos\_data, }\StringTok{"full\_biomodelos\_data.csv"}\NormalTok{)}
\end{Highlighting}
\end{Shaded}

Count records for each species

\begin{Shaded}
\begin{Highlighting}[]
\NormalTok{species\_num\_final }\OtherTok{\textless{}{-}}\NormalTok{ full\_biomodelos\_data }\SpecialCharTok{\%\textgreater{}\%}
  \FunctionTok{count}\NormalTok{(scientific\_name) }

\FunctionTok{tabyl}\NormalTok{(full\_biomodelos\_data, scientific\_name)}
\end{Highlighting}
\end{Shaded}


\end{document}
